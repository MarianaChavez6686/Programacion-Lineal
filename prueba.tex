\documentclass{article}

\usepackage[utf8]{inputenc}

\usepackage{amsmath}

\usepackage[spanish]{babel}

\title{Apuntes de programación lineal}

\author{Rodolfo Chino Mariana Chávez}


\begin{document}

\maketitle
\tableofcontents

\section{Introducción}

\label{sec:introduccion}


La forma estándar de un problema de programación lineal es:
Dados la matriz $A$ y vectores $b,c$, maximizar $c^tx$ sujeto a
$Ax\leq b$.

\subsection{Ejercicios}
\label{sec:ejercicios}

\subsubsection{}
Un gerente está planeando cómo distribuir la producción de dos productos entre dos máquinas. Para ser manufacturado cada producto requiere cierto tiempo (en horas) en cada una de las máquinas.

El tiempo requerido está resumido en la siguiente tabla:\\
\medskip
\begin{tabular}{|c|c|c|}
\hline
Producto/Máquina &	1 &	2\\
\hline
A                &  	1 &	1\\
\hline
B                &   	2 &	1\\                          
\hline 
\end{tabular}
\bigskip

La máquina 1 está disponible 40 horas a la semana y la 2 está disponible 34 horas a la semana.

Si la utilidad obtenida al vender los productos A y B es de 2, 3 pesos
por unidad, respectivamente, ¿cuál debe ser la producción semanal que
maximiza la utilidad? ¿Cuál es la utilidad máxima?


\subsubsection{}

\begin{equation*}
  \label{eq:1}
  A=
  \begin{pmatrix}
    0 & 1 & 2\\
    3 & -1 & 5
  \end{pmatrix}
  \begin{pmatrix}
    2 & 0\\
    7 & 1
  \end{pmatrix}
\end{equation*}


\end{document}
