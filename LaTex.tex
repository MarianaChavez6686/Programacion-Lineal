\documentclass{article}

\usepackage[spanish]{babel}
\usepackage{amsmath}
\usepackage[utf8]{inputenc}

\title{Método simplex}
\author{Chino Chávez}
\begin{document}

\maketitle
\section{Introducción}
\label{sec:introduccion}

El método simplex es un algoritmo para resolver problemas de
programación lineal. Fue inventado por el matemático George Dantzing
en el año 1947
\section{Ejemplo}
\label{sec:ejemplo}

Ilustraremos la aplicacion del método simplex con un ejemplo:

\begin{equation*}
  \begin{aligned}
\text{Maximizar} \quad & 2x+2y\\
\text{Sujeto a} \quad &
   \begin{aligned}
2x+y &\leq 4\\
x+2y &\leq 5\\
-2x+y &\geq 2\\
x,y &\geq 0
   \end{aligned}
\end{aligned}
\end{equation*}
Se puede observar que este problema no esta en forma éstandar por lo
que lo primero que haremos es tenerlo en esa forma, para ello debemos
poner todas las ecuaciones con $\leq$.
\begin{equation*}
  \begin{aligned}
\text{Maximizar} \quad & 2x+2y\\
\text{Sujeto a} \quad &
   \begin{aligned}
2x+y &\leq 4\\
x+2y &\leq 5\\
2x-y &\leq -2\\
x,y &\geq 0
   \end{aligned}
\end{aligned}
\end{equation*}
Ahora para igualar las desigualdades colocaremos las variables de
holgura,entonces el ploblema sería de la siguiente forma.
\begin{equation*}
  \begin{aligned}
\text{Maximizar} \quad & 2x+2y\\
\text{Sujeto a} \quad &
   \begin{aligned}
     2x+y+z\phantom{+w}\phantom{+v} &= 4\\
x+2y\phantom{+z}+w\phantom{+v} &= 5\\
2x-y\phantom{+z}\phantom{+w}+v &= -2\\
x,y,z,w,v &\geq 0
 \end{aligned}
\end{aligned}
\end{equation*}
\end{document}
